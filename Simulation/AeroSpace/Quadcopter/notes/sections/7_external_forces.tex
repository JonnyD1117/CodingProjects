\newpage
\section{External \& Aerodynamic Forces/Torques and Actuator Dynamics}

With the Newton-Euler equations defined as simplified as much as possible, the next step is to determine the external forces and torques acting on the system that are unique to the application (e.g. quadcopter dynamics). As with most dynamic simulations, these external forces and torques are either do not fully encapsulate the complete dynamics of the systems or are intentionally simplified.

\subsection{External \& Aerodynamics Forces}

\subsubsection{Force Due to Gravity}
The force of gravity is relative to the inertial reference frame and is proportional to the mass of a system by the gravitational constant.

$$F_{gravity} \biggr\rvert_{inertial} = \bvect{0}{0}{-mg} $$

In the body frame... 
$$\boxed{F_{gravity} \biggr\rvert_{body} = R_{zyx}(\phi, \theta, \psi)\biggr\rvert_{I}^{B}\bvect{0}{0}{-mg}} $$


\subsubsection{Net Thrust Force}
From Conservation of Energy and Blade Momentum Theory it can be determined that the thrust force generated by a single rotating propellor is... 

$$f_{thrust_i}\biggr\rvert_{body} = \left( \frac{K_{v}K_{\tau}\sqrt{2\rho A}}{K_{t}}\omega \right)^{2}  \quad \quad \rightarrow \quad \quad f_{thrust_i}\biggr\rvert_{body} = k_{lift}\cdot\omega_{i}^{2}$$

When considering the net thrust generated by the four propellors of a quadcopter, (assuming the propellors and motors driving them are identical) we can say... 


$$F_{thrust}\biggr \rvert_{body} = \sum_{i=1}^{4} f_{thrust_i}\quad \quad \rightarrow \boxed{F_{thrust}\biggr \rvert_{body} = k_{lift}\bvect{0}{0}{\sum\omega_{i}^{2}}}$$



\subsubsection{Drag Force}

There are many different types of drag forces that can be modeled for the quadcopter; however for simplicity we are only going to cover a "general" drag term that is proportional to the velocity of the vehicle. The standard force of drag due to fluid friction is written below. 

$$F_{drag}\biggr \rvert_{inertial} = \frac{1}{2}\rho C_{D} A \vect{v}^2$$

By collapsing all constants into a single term, the drag force can be expressed as ... 

$$ \boxed{F_{drag}\biggr \rvert_{inertial} = \bvect{-k_d\dot{x}}{-k_d\dot{y}}{-k_d\dot{z}}}$$

\subsubsection{Total External Forces}

$$\sum F_{ext} = F_{gravity} \biggr\rvert_{body} + F_{thrust}\biggr \rvert_{body} F_{thrust}\biggr \rvert_{body}$$

$$\boxed{\sum F_{ext} \biggr\rvert_{body}= R_{zyx}(\phi, \theta, \psi)\biggr\rvert_{I}^{B}\left(\bvect{0}{0}{-mg} + \bvect{-k_d\dot{x}}{-k_d\dot{y}}{-k_d\dot{z}}\right) + k_{lift}\bvect{0}{0}{\sum\omega_{i}^{2}} }$$


\subsection{External \& Aerodynamic Torques}

Modeling torques is important to study and simulation of quadcopters as controlling the balance of torque in the system is the only means of translational and rotational control that is available to a quadcopter. This is accomplished by combining torques about the CG of the quadcopter. 

\subsubsection{Torque Due to Drag}

According to $F_{drag} \rvert_{inertial} = \frac{1}{2}\rho C_{D} A \vect{v}^2$, there is a force due to drag; however, if this forces line of action is acting at some distance $r$ from the center of gravity of the quadcopter, there is a corresponding torque to the action of this force. 
$$\tau_{drag} = \frac{1}{2}\rho C_{D} A (r\perp\omega)^2 \quad \quad where \quad  V = \vect{r} \times \vect{\omega}$$

Again, collapsing the constant terms gives us... 
$$\tau_{drag} = b\cdot \omega^2$$


\subsubsection{Motor Torques}

Several torques are generated by the motor, including torques due to drag, inertial effects, and gyroscopic effects. In this paper, we will ignore both the gyroscopic and inertial torques.

$$\tau_{M_{i}} = b\cdot\omega_{i}^{2} + \Gamma_{gyro} +I_{M}\dot{\omega_{i}}$$


In addition to the simplifications described above, since quadcopters spin two propellers clockwise and the other two propellers counter-clockwise, there is a size change for the direction of the angular velocity of the propeller depending on which propeller is being observed. This can be captured with the following equation. 
$$\tau_{M_{i}} = (-1)^{i+1} b\cdot\omega_{i}^{2}$$

\subsubsection{Yaw Torque}
The formulation for the total torque around the yaw axis is described as the sum of all motor torques (accounting for sign changes)

$$\tau_{\psi} = b(\omega_{1}^{2}- \omega_{2}^{2} + \omega_{3}^{2} - \omega_{4}^{2})$$

\subsubsection{Roll Torque}
The roll torque about the CG is a product of two motors (about the roll axis) mismatching torques and producing a net torque about that axis. 


$$\tau_{\phi} = \sum r \times F_{thrust} = Lk(\omega_{1}^{2} - \omega_{3}^{2})$$


\subsubsection{Pitch Torque}
As will roll torques, pitch torques work the same but around the pitch axis of the body.

$$\tau_{\theta} = \sum r \times F_{thrust} = Lk(\omega_{2}^{2} - \omega_{4}^{2})$$

\subsection{Total Body Torques}

All of the torques acting on the body and be combined to form the following expression for the total torque acting on the body. 

$$\boxed{\tau_{body} = \bvect{\tau_{\phi}}{\tau_{\theta}}{\tau_{\psi}} = \bvect{Lk(\omega_{1}^{2} - \omega_{3}^{2})}{Lk(\omega_{2}^{2} - \omega_{4}^{2})}{b(\omega_{1}^{2}- \omega_{2}^{2} + \omega_{3}^{2} - \omega_{4}^{2})}}$$