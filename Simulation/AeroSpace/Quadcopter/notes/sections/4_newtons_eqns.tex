\newpage 

\section{Newtons Equations}

According to Newton's Second Law of Motion, the time rate of change of the linear momentum is equal to the sum of external forces on a dynamic system in translation. 

$$\sum \vect{F} = \dot{\vect{L}} \quad \text{Where} \quad \vect{L} = m\vect{V}$$ 
$$\sum\vect{F} = \dot{m\vect{V}} = m\dot{\vect{V}} + \dot{m}\vect{V}$$

When taken in a rotating reference frame... 
$$\frac{d(\vect{V})}{dt} \biggr \rvert_{inertial} = \dot{\vect{V}} \biggr \rvert_{body} + \omega \times \vect{V}$$

This equation allows us to express the derivatives in terms of the velocity of the body frame and the angular velocity of the body frame with respect to the inertial frame.

$$\sum\vect{F} = m(\dot{\vect{V}} + \omega \times \vect{V}) + \dot{m}\vect{V}$$

If the rate of change of mass is zero or negligible then... 
$$\sum\vect{F} = m\dot{V} + \dot{m}\vect{V} = m \vect{a}$$

Leaving the expression 
$$\sum\vect{F} =  m(\dot{\vect{V}} + \omega \times \vect{V})$$


